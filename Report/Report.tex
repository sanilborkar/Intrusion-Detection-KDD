\documentclass[11pt]{article}

\usepackage[parfill]{parskip}
\usepackage{listings}
\usepackage{fancyvrb}
\usepackage{lipsum}
\usepackage{mathtools}
\usepackage{graphicx}

\title{\textbf{Intrusion Detection Using KDD Dataset}}
\author{Divya Chaganti\\
		Saili Sahasrabuddhe\\
		Ashwin Viswanathan\\
		Vignesh Miriyala\\
		Sanil Sinai Borkar}
\date{December 16, 2015}

\begin{document}

\maketitle

% ------------------------------------------ Introduction --------------------------------------------
\section{Introduction}
Our project is to use a data mining approach to develop an intrusion detection system (IDS) using the KDD dataset~\cite{kdddata}.

% ------------------------------------------ Feature Extraction --------------------------------------------
\section{Feature Extraction}
The original KDD dataset contains 42 features including the label. Since working this size of data may not be feasible always from the perspective of a real-time IDS, we had to reduce the number of features used. This dimensionality reduction was achieved using Principal Component Analysis (PCA).

34 of the 42 features of the dataset were continuous on which PCA was applied. The result of applying PCA is placed in the file {\it PCA\_Result.txt}.

Out of all the 34 continuous features, we picked up the principal components that contained positive weights for most of the {\it error rate} indicating features. Out of the 34 candidate principal components, we narrowed down it to 2 principal components, and selected the one that had more positive weights. This led us to finalize our principal component which is specified below:

$Comp 23 = 0.124*num\_access\_files - 0.131*serror\_rate + 0.141*srv\_serror\_rate - 0.490*srv\_rerror\_rate + 0.107*same\_srv\_rate + 0.126*diff\_srv\_rate - 0.154*dst\_host\_diff\_srv\_rate + \\ 0.263*dst\_host\_srv\_diff\_host\_rate - 0.122*dst\_host\_serror\_rate + 0.162*dst\_host\_srv\_serror\_rate + 0.717*dst\_host\_rerror\_rate$

Based on this principal component, 11 features were selected out of the available 41 as shown in Table~\ref{Tab:features} as per the description given in \cite{kddnames}.

\begin{table}
\caption{Final Feature Set}
\centering
\begin{tabular}{l*{2}{l}}
Feature & Description \\
\hline
num\_access\_files & number of operations on access control files  \\
serror\_rate & \% of connections that have ``SYN'' errors  \\
srv\_serror\_rate & \% of connections that have ``SYN'' errors  \\
srv\_rerror\_rate & \% of connections that have ``REJ'' errors  \\
same\_srv\_rate & \% of connections to the same service \\
diff\_srv\_rate & \% of connections to different services \\
dst\_host\_diff\_srv\_rate &  \\
dst\_host\_srv\_diff\_host\_rate &  \\
dst\_host\_serror\_rate &  \\
dst\_host\_srv\_serror\_rate &  \\
dst\_host\_rerror\_rate &  \\
\end{tabular}
\label{Tab:features}
\end{table}

Since the IDS needs to tag each network packet as either malicious (attack packet) or benign (normal packet), the type of attack is not significant but only the presence of an attack packet is. Therefore, as a pre-processing step, the normal packets were labeled as `normal' and everything else was tagged as an `attack' packet.

% ------------------------------------------ Classification --------------------------------------------
\section{Classification}
Decision Tree classifier was used to classify the instances. Out of the 494022 packets, 80\% of it was used as training data (395218 records), and the remaining 20\% was used for testing (98803 records).

The time taken to train a decision tree classifier model was around {\bf 14 seconds} on an average, and accuracy was {\bf 95.5318\%}. The summary of the classifier model is given below:

\begin{Verbatim}[fontsize=\small]
=== Summary ===

Correctly Classified Instances      377559               95.5318 %
Incorrectly Classified Instances     17659                4.4682 %
Kappa statistic                          0.8471
Mean absolute error                      0.083 
Root mean squared error                  0.2037
Relative absolute error                 26.2432 %
Root relative squared error             51.2282 %
Coverage of cases (0.95 level)          99.8133 %
Mean rel. region size (0.95 level)      81.0894 %
Total Number of Instances           395218     

=== Confusion Matrix ===

      a      b   <-- classified as
 316384   1011 |      a = attack.
  16648  61175 |      b = normal.
\end{Verbatim}

The prediction for the 98803 records was done in {\bf 0.2 seconds} with an accuracy of {\bf 95.46\%}. The confusion matrix for the prediction is given below:

\begin{Verbatim}[fontsize=\small]
Confusion Matrix and Statistics

          Reference
Prediction attack. normal.
   attack.   79084    4223
   normal.     264   15232
                                          
               Accuracy : 0.9546          
                 95% CI : (0.9533, 0.9559)
    No Information Rate : 0.8031          
    P-Value [Acc > NIR] : < 2.2e-16       
                                          
                  Kappa : 0.8445          
 Mcnemar's Test P-Value : < 2.2e-16       
                                          
            Sensitivity : 0.9967          
            Specificity : 0.7829          
         Pos Pred Value : 0.9493          
         Neg Pred Value : 0.9830          
             Prevalence : 0.8031          
         Detection Rate : 0.8004          
   Detection Prevalence : 0.8432          
      Balanced Accuracy : 0.8898          
                                          
       'Positive' Class : attack.  
\end{Verbatim}      

% ------------------------------------------ Association --------------------------------------------
\section{Association}


% ------------------------------------------ Clustering --------------------------------------------
\section{Clustering}


% ------------------------------------------ Conclusion --------------------------------------------
\section{Conclusion}


% ------------------------------------------ References --------------------------------------------
\bibliographystyle{IEEEtran}
\bibliography{Reference}

\end{document}